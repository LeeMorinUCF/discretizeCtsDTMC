\documentclass[article]{jss}

%% -- LaTeX packages and custom commands ---------------------------------------

%% recommended packages
\usepackage{thumbpdf,lmodern}

%% another package (only for this demo article)
\usepackage{framed}

%% new custom commands
\newcommand{\class}[1]{`\code{#1}'}
\newcommand{\fct}[1]{\code{#1()}}


%% -- Article metainformation (author, title, ...) -----------------------------

%% - \author{} with primary affiliation
%% - \Plainauthor{} without affiliations
%% - Separate authors by \And or \AND (in \author) or by comma (in \Plainauthor).
%% - \AND starts a new line, \And does not.
\author{Lealand Morin\\University of Central Florida
   \And Harry J. Paarsch\\University of Central Florida}
\Plainauthor{Lealand Morin, Harry J. Paarsch}

%% - \title{} in title case
%% - \Plaintitle{} without LaTeX markup (if any)
%% - \Shorttitle{} with LaTeX markup (if any), used as running title
\title{Intervention analysis using the cross section of a wide panel}
\Plaintitle{Intervention analysis using the cross section of a wide panel}
\Shorttitle{Intervention analysis using the cross section}

%% - \Abstract{} almost as usual
\Abstract{
   \pkg{interCross} uses the variation of successive observations in the
    cross section of a wide panel to identify the law of motion 
    for conduction an intervention analysis.
    \pkg{interCross} creates a discrete approximation to a low-order Markov
    process defined on a continuous state space in discrete time.
    Once the state space is discretized, \pkg{interCross} provides tools
    to estimate the transition matrices and analyze the Markov process. 
}

%% - \Keywords{} with LaTeX markup, at least one required
%% - \Plainkeywords{} without LaTeX markup (if necessary)
%% - Should be comma-separated and in sentence case.
\Keywords{Intervention analysis, wide panel, Markov chain, Markov process, discretization}
\Plainkeywords{Intervention analysis, wide panel, Markov chain, Markov process, discretization}

%% - \Address{} of at least one author
%% - May contain multiple affiliations for each author
%%   (in extra lines, separated by \emph{and}\\).
%% - May contain multiple authors for the same affiliation
%%   (in the same first line, separated by comma).

\Address{
  Lealand Morin\\
  Department of Economics\\
  University of Central Florida\\
  E-mail: \email{Lealand.Morin@ucf.edu} \\
  \phantom{0}\\
  Harry J. Paarsch\\
  Department of Economics\\
  University of Central Florida\\
  E-mail: \email{Harry.Paarsch@ucf.edu}
}

\begin{document}


%% -- Introduction -------------------------------------------------------------

%% - In principle "as usual".
%% - But should typically have some discussion of both _software_ and _methods_.
%% - Use \proglang{}, \pkg{}, and \code{} markup throughout the manuscript.
%% - If such markup is in (sub)section titles, a plain text version has to be
%%   added as well.
%% - All software mentioned should be properly \cite-d.
%% - All abbreviations should be introduced.
%% - Unless the expansions of abbreviations are proper names (like "Journal
%%   of Statistical Software" above) they should be in sentence case (like
%%   "generalized linear models" below).


\section[Introduction]{Introduction} \label{sec:intro}


\begin{leftbar}
The introduction is in principle ``as usual''. However, it should usually embed
both the implemented \emph{methods} and the \emph{software} into the respective
relevant literature. For the latter both competing and complementary software
should be discussed (within the same software environment and beyond), bringing
out relative (dis)advantages. All software mentioned should be properly
\verb|\cite{}|d. (See also Appendix~\ref{app:bibtex} for more details on
\textsc{Bib}{\TeX}.)

For writing about software JSS requires authors to use the markup
\verb|\proglang{}| (programming languages and large programmable systems),
\verb|\pkg{}| (software packages), \verb|\code{}| (functions, commands,
arguments, etc.). If there is such markup in (sub)section titles (as above), a
plain text version has to be provided in the {\LaTeX} command as well. Below we
also illustrate how abbrevations should be introduced and citation commands can
be employed. See the {\LaTeX} code for more details.
\end{leftbar}


This article illustrates how to
uss the variation of successive observations in the
    cross section of a wide panel to identify the law of motion 
    for conduction an intervention analysis.
The \pkg{interCross} package creates a discrete approximation to
the distribution function of a population of individuals following 
   a Markov  process defined on a continuous state space in discrete time.
  Once the state space is discretized, \pkg{interCross} provides tools
  to estimate the transition matrices and analyze the Markov process.
  It is used to model a population of individuals, each following a  
  continuous-state Markov process in discrete time.
% 


Describe intervention analysis and survey R packages.

% 
There are several R packages \citep{R} available for working with Markov processes. 
% 
The packages \pkg{markovchain} \citep{pkg:markovchain} and \pkg{DTMCPack} \citep{pkg:DTMCPack} provide tools for basic computations with Markov chains.  
% 
The \pkg{mcmc} is designed for working with Monte Carlo Markov Chains. 
% 
Packages \pkg{HMM} \citep{pkg:mcmc} and \pkg{depmixS4} \citep{pkg:depmixS4} are designed for fitting Hidden Markov Models. 
% (We should explain how what we do is different than Hidden Markov Models.) 
%

There are several packages aimed at specific applications of Markov chains. 
% 
Some of these include packges designed for application in health care. 
There is \pkg{TPmsm} \citep{pkg:TPmsm} for estimating transition probabilities for 3-state progressive disease models and \pkg{heemod} \citep{pkg:heemod} for applying Markov models to health care economic applications. 
% 
Aimed at specific applications, some of these packages assume a considerable knowledge of the relevant subject matter and theory behind those applications. 
% 

The packages \pkg{msm} \citep{pkg:msm} and \pkg{SemiMarkov} \citep{pkg:SemiMarkov} are used for fitting multistate models to panel data, along with \pkg{mstate} \citep{pkg:mstate} for survival analysis applications. 
These packages are designed to model discrete-state Markov processes in continuous time, with transitions taking place at a stochastic arrival times. 



\section[Model]{Model} \label{sec:model}


Describe model here.


%% -- Manuscript ---------------------------------------------------------------

%% - In principle "as usual" again.
%% - When using equations (e.g., {equation}, {eqnarray}, {align}, etc.
%%   avoid empty lines before and after the equation (which would signal a new
%%   paragraph.
%% - When describing longer chunks of code that are _not_ meant for execution
%%   (e.g., a function synopsis or list of arguments), the environment {Code}
%%   is recommended. Alternatively, a plain {verbatim} can also be used.
%%   (For executed code see the next section.)


\begin{leftbar}
Note that around the \verb|{equation}| above there should be no spaces (avoided
in the {\LaTeX} code by \verb|%| lines) so that ``normal'' spacing is used and
not a new paragraph started.
\end{leftbar}


%% -- Illustrations ------------------------------------------------------------

%% - Virtually all JSS manuscripts list source code along with the generated
%%   output. The style files provide dedicated environments for this.
%% - In R, the environments {Sinput} and {Soutput} - as produced by Sweave() or
%%   or knitr using the render_sweave() hook - are used (without the need to
%%   load Sweave.sty).
%% - Equivalently, {CodeInput} and {CodeOutput} can be used.
%% - The code input should use "the usual" command prompt in the respective
%%   software system.
%% - For R code, the prompt "R> " should be used with "+  " as the
%%   continuation prompt.
%% - Comments within the code chunks should be avoided - these should be made
%%   within the regular LaTeX text.

\section{Illustrations} \label{sec:illustrations}

A demonstration of analysis is shown in \verb|interCross_demo.R| and it serves as an example of what a typical session of model specification, estimation and testing can include. This procedure includes the following steps:

\begin{enumerate}
\item Organizing data
\item Choosing estimation options
\item Lag selection
\item Model estimation
\item Hypothesis testing
\end{enumerate}


\subsection{Organizing data} \label{subsec:data}


\subsection{Choosing options} \label{subsec:options}


\subsection{Lag-order selection}  \label{subsec:lags}


\subsection{Model estimation}  \label{subsec:estimation}


\subsection{Hypothesis testing} \label{subsec:testing}



%
\begin{CodeChunk}
\begin{CodeInput}
R> data("quine", package = "MASS")
\end{CodeInput}
\end{CodeChunk}
%


\begin{leftbar}
For code input and output, the style files provide dedicated environments.
Either the ``agnostic'' \verb|{CodeInput}| and \verb|{CodeOutput}| can be used
or, equivalently, the environments \verb|{Sinput}| and \verb|{Soutput}| as
produced by \fct{Sweave} or \pkg{knitr} when using the \code{render_sweave()}
hook. Please make sure that all code is properly spaced, e.g., using
\code{y = a + b * x} and \emph{not} \code{y=a+b*x}. Moreover, code input should
use ``the usual'' command prompt in the respective software system. For
\proglang{R} code, the prompt \code{"R> "} should be used with \code{"+  "} as
the continuation prompt. Generally, comments within the code chunks should be
avoided -- and made in the regular {\LaTeX} text instead. Finally, empty lines
before and after code input/output should be avoided (see above).
\end{leftbar}


\begin{figure}[h]
  \centering
  \includegraphics[scale = .6, keepaspectratio=true]{Figures/Rlogo.png}
  \caption{Caption goes here}
  \label{fig:Rlogo}
\end{figure}


%% -- Summary/conclusions/discussion -------------------------------------------

\section{Summary and discussion} \label{sec:summary}

\begin{leftbar}
As usual \dots
\end{leftbar}


%% -- Optional special unnumbered sections -------------------------------------

\section*{Computational details}

\begin{leftbar}
If necessary or useful, information about certain computational details
such as version numbers, operating systems, or compilers could be included
in an unnumbered section. Also, auxiliary packages (say, for visualizations,
maps, tables, \dots) that are not cited in the main text can be credited here.
\end{leftbar}

The results in this paper were obtained using
\proglang{R}~4.0.2. 
with the \pkg{interCross} package Version 0.0.0.9000. 
\proglang{R} itself
and all packages used are available from the Comprehensive
\proglang{R} Archive Network (CRAN) at
\url{https://CRAN.R-project.org/}.


\section*{Acknowledgments}

\begin{leftbar}
All acknowledgments (note the AE spelling) should be collected in this
unnumbered section before the references. It may contain the usual information
about funding and feedback from colleagues/reviewers/etc. Furthermore,
information such as relative contributions of the authors may be added here
(if any).
\end{leftbar}


%% -- Bibliography -------------------------------------------------------------
%% - References need to be provided in a .bib BibTeX database.
%% - All references should be made with \cite, \citet, \citep, \citealp etc.
%%   (and never hard-coded). See the FAQ for details.
%% - JSS-specific markup (\proglang, \pkg, \code) should be used in the .bib.
%% - Titles in the .bib should be in title case.
%% - DOIs should be included where available.

\bibliography{references}


%% -- Appendix (if any) --------------------------------------------------------
%% - After the bibliography with page break.
%% - With proper section titles and _not_ just "Appendix".

\newpage

\begin{appendix}

\section{More technical details} \label{app:technical}

\begin{leftbar}
Appendices can be included after the bibliography (with a page break). Each
section within the appendix should have a proper section title (rather than
just \emph{Appendix}).

For more technical style details, please check out JSS's style FAQ at
\url{https://www.jstatsoft.org/pages/view/style#frequently-asked-questions}
which includes the following topics:
\begin{itemize}
  \item Title vs.\ sentence case.
  \item Graphics formatting.
  \item Naming conventions.
  \item Turning JSS manuscripts into \proglang{R} package vignettes.
  \item Trouble shooting.
  \item Many other potentially helpful details\dots
\end{itemize}
\end{leftbar}


\section[Using BibTeX]{Using \textsc{Bib}{\TeX}} \label{app:bibtex}

\begin{leftbar}
References need to be provided in a \textsc{Bib}{\TeX} file (\code{.bib}). All
references should be made with \verb|\cite|, \verb|\citet|, \verb|\citep|,
\verb|\citealp| etc.\ (and never hard-coded). This commands yield different
formats of author-year citations and allow to include additional details (e.g.,
pages, chapters, \dots) in brackets. In case you are not familiar with these
commands see the JSS style FAQ for details.

Cleaning up \textsc{Bib}{\TeX} files is a somewhat tedious task -- especially
when acquiring the entries automatically from mixed online sources. However,
it is important that informations are complete and presented in a consistent
style to avoid confusions. JSS requires the following format.
\begin{itemize}
  \item JSS-specific markup (\verb|\proglang|, \verb|\pkg|, \verb|\code|) should
    be used in the references.
  \item Titles should be in title case.
  \item Journal titles should not be abbreviated and in title case.
  \item DOIs should be included where available.
  \item Software should be properly cited as well. For \proglang{R} packages
    \code{citation("pkgname")} typically provides a good starting point.
\end{itemize}
\end{leftbar}

\end{appendix}

%% -----------------------------------------------------------------------------


\end{document}
